\documentclass[a4paper]{ltxdoc}

\usepackage{holtxdoc}
\usepackage{url}
\usepackage{booktabs}
\usepackage{listings}
\usepackage{tikz}
\hypersetup{%
  plainpages=false,%
  bookmarksnumbered,%
  pdftitle={The quran-en Package},%
  pdfkeywords={quran-en, surah, ayah, juz, hizb, ruku, manzil},%
  pdfauthor={Seiied-Mohammad-Javad Razavian},%
  baseurl={http://mirrors.ctan.org/macros/xetex/latex/quran-en/doc/quran-en-doc.pdf},%
}
\usepackage[all]{quran-en}
% because of definition of \XeTeX and \XeLaTeX symbols in bidi,
% I should undef these macro that are also defined in holtxdoc package.
\makeatletter
\bidi@undef\XeTeX
\bidi@undef\XeLaTeX
\makeatother
\usepackage{arabxetex}
\makeatletter
\bidi@BeforeBeginEnvironment{declcs}{\vspace*{-4mm}}
\makeatother

%%%%%%%%%%%%%%%%%%%%%%%%%%%%%%%%%%%%%%%%%%%%%

\def\boxcs#1{\leftline{\fbox{\mcs{#1}}}}
\def\mcs#1{\lr{\texttt{\textbackslash{}#1}}}
\def\tt#1{\lr{\texttt{#1}}}
\lstdefinestyle{BashInputStyle}{
  language=bash,
  basicstyle=\footnotesize\sffamily,
  frame=tb,
  columns=fullflexible,
  backgroundcolor=\color{gray!10},
}

% Define box and box title style
\tikzstyle{mybox} = [draw=black, fill=gray!20, very thick,
    rectangle, rounded corners, inner sep=10pt, inner ysep=20pt]
\tikzstyle{fancytitle} =[fill=gray, text=white]

\def\mx#1#2{\mybox{#1}{#2}{.45\textwidth}}
\def\mxf#1#2{\mybox{#1}{#2}{\textwidth}}

\def\mybox#1#2#3{
\begin{tikzpicture}
\node [mybox] (box){%
    \begin{minipage}[t]{#3}
        #2
    \end{minipage}
};
\node[fancytitle, anchor=west, right=10pt] at (box.north west) {\small \cs{#1}};
\node (hole) [anchor=north east, left=5pt ] at (box.north east) { \tikz\fill[very thick,white] (0,0) circle (12pt); };
\node[ ] at (hole.center) {\includegraphics[width=.05\textwidth]{quran.png}};
\end{tikzpicture}%
}

%%%%%%%%%%%%%%%%%%%%%%%%%%%%%%%%%%%%%%%%%%%%%
\title{\includegraphics[scale=.3]{quran.png}\\
The \xpackage{quran-en} Package\footnote{To one having devoted his life to the holy Quran}\\
}
\author{Seiied Mohammad Javad Razavian\\\xemail{javadr@gmail.com}}

\date{\quranendate,  Version \quranenversion\footnote{First release: November 4th, 2023}}

\begin{document}
\maketitle

\tableofcontents
%\newpage

\section{Introduction}
The \xpackage{quran} package provides several macros for
typesetting the whole or any parts of the holy Quran based on its popular divisions. That package also develops
commands for some translations of the holy Quran including German, Engilsh, and Persian.
Some people asked me to include the other translations into the \xpackage{quran} package,
but because of some technical issues I decided to create new packages for other translations.
One of these variations is \xpackage{quran-en} including several English translations of the holy Quran
provided by \url{tanzil.net}, i.e. \emph{``ahmedali''}, \emph{``ahmedraza''}, \emph{``arberry''}, \emph{``daryabadi''}, \emph{``hilali''}, \emph{``itani''}, \emph{``maududi''}, \emph{``mubarakpuri''}, \emph{``pickthall''}, \emph{``qarai''}, \emph{``qaribullah''}, \emph{``sahih''}, \emph{``sarwar''}, \emph{``shakir''}, \emph{``wahiduddin''}, and \emph{``yusufali''}.


\section{Loading The Package}
The package will be loaded in the ordinary way
\cs{usepackage[option]\{quran-en\}}.
After loading the package, it writes some information about itself to the
terminal and to the log file, too. The information is something like:

\begin{quote}
\begin{lstlisting}[style=BashInputStyle, language=tex, escapechar={|}]
Package: quran-id |\quranendate| v|\quranenversion|
English translation extension to the quran package.
\end{lstlisting}
\end{quote}


\section{Options of The Package}\label{sec:qurantypesetting}
There are several options by the names of \xoption{ahmedali}, \xoption{ahmedraza}, \xoption{arberry}, \xoption{daryabadi}, \xoption{hilali}, \xoption{itani}, \xoption{maududi}, \xoption{mubarakpuri}, \xoption{pickthall}, \xoption{qarai}, \xoption{qaribullah}, \xoption{sahih}, \xoption{sarwar}, \xoption{shakir}, \xoption{wahiduddin}, and \xoption{yusufali}.%
%
\foreach\x in {ahmedali,ahmedraza,arberry,daryabadi,hilali,itani,maududi,mubarakpuri,pickthall,qarai,qaribullah,sahih,sarwar,shakir,wahiduddin,yusufali,all}{%
\marginpar{\xoption{\x}\smallskip}%
}
%
If you pass any combinations of these options, you will be enabled to typeset these translations in a way
that the \xpackage{quran} package does.
There is also \xoption{all} option which loads all English translations provided by the \xpackage{quran-en} package.
The package loads \xoption{ahmedali} option by default.

This package is completely built on top of the \xpackage{quran} package, therefore you can pass every options
defined in that package to the \xpackage{quran-en} package.

\section{Differences between \xpackage{quran} and \xpackage{quran-en}}
At first glance, \xpackage{quran-en}  offers all functionalities of \xpackage{quran}. Therefore,
both packages are the same in this manner except one difference. The \xpackage{quran-en} extends the \xpackage{quran}
by adding 16 other English translations, namely `ahmedali', `ahmedraza', `arberry', `daryabadi', `hilali', `itani', `maududi', `mubarakpuri', `pickthall', `qarai', `qaribullah', `sahih', `sarwar', `shakir', `wahiduddin', and `yusufali'. All English translations
provided by the package are shown in table~\ref{tab:idtrans}.

\begin{table}[!htbp]
\centering
\begin{tabular}{|c|l|c|}
    \toprule
    order & \multicolumn{1}{c|}{translator} & option's name \\\midrule
    1 & Ahmed Ali & ahmedali \\
    2 & Ahmed Raza Khan & ahmedraza \\
    3 & A. J. Arberry & arberry \\
    4 & Abdul Majid Daryabadi & daryabadi \\
    5 & Muhammad Taqi-ud-Din al-Hilali and Muhammad Muhsin Khan & hilali \\
    6 & Talal Itani & itani \\
    7 & Abul Ala Maududi & maududi \\
    8 & Safi-ur-Rahman al-Mubarakpuri & mubarakpuri \\
    9 & Mohammed Marmaduke William Pickthall & pickthall \\
    10 & Ali Quli Qarai & qarai \\
    11 & Hasan al-Fatih Qaribullah and Ahmad Darwish & qaribullah \\
    12 & Saheeh International & sahih \\
    13 & Muhammad Sarwar & sarwar \\
    14 & Mohammad Habib Shakir & shakir \\
    15 & Wahiduddin Khan & wahiduddin \\
    16 & Abdullah Yusuf Ali & yusufali \\
    \bottomrule
\end{tabular}
    \caption{English translations used in the package and their option's names}
    \label{tab:idtrans}
\end{table}

\subsection{How to Change the Current English Translation}
    The \xpackage{quran-en} package uses ahmedali's translation by default.
    The following macro could be used to change the default English translation.
    \begin{declcs}{enSetTrans}\marg{index}
    \end{declcs}
    The \meta{index} could be an integer or a name; both ``order'' and ``option's name''
    shown in Table~\ref{tab:idtrans} are appliable. Both \cs{enSetTrans\{2\}} and \cs{enSetTrans\{ahmedraza\}}, for example,
    have the same effect.

\subsection{How to Get the Name of Current Indonesian Translation}
    \begin{declcs}{idGetTrans}
    \end{declcs}
    The above macro returns the name of current Indonesian translation, i.e.  one of ``ahmedali'', ``ahmedraza'', ``arberry'', ``daryabadi'', ``hilali'', ``itani'', ``maududi'', ``mubarakpuri'', ``pickthall'', ``qarai'', ``qaribullah'', ``sahih'', ``sarwar'', ``shakir'', ``wahiduddin'', or ``yusufali''.


\subsection{How to Typeset the Indonesian Translation}
    It's completely similar to the \xpackage{quran} package.
    All the following macros are usable.

\begin{multicols}{2}
    \begin{itemize}
        \item \cs{quransurahen}
        \item \cs{quranayahen}
        \item \cs{quranpageen}
        \item \cs{quranjuzen}
        \item \cs{quranhizben}
        \item \cs{quranquarteren}
        \item \cs{quranrukuen}
        \item \cs{quranmanzilen}
        \item \cs{qurantexten}
    \end{itemize}
\end{multicols}

    \centerline{\mxf{quransurah*}
    {\begin{arab}\small\quransurah*\end{arab}}}

    \foreach \x/\y in {ahmedali/1,ahmedraza/2,arberry/3,daryabadi/4,hilali/5,itani/6,maududi/7,mubarakpuri/8,pickthall/9,qarai/10,qaribullah/11,sahih/12,sarwar/13,shakir/14,wahiduddin/15,yusufali/16} {%
        \centerline{\mxf{enSetTrans\{\x\}\textbackslash{}quransurahen*}
        {\enSetTrans{\y}\quransurahen*}}
    }

\end{document}

